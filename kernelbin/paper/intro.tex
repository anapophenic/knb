\section{Introduction Outline}

We work with DNA methylation data from brain cells.  The data in each location is contains coverage and methylation counts.

Our goal is to learn the states in each location, representing functional regions. The states are summarized as its methylation probability.

Differential Methylation States: by working on aligned methylation sequences across different cell types, we aim to identify
states that corresponds to differential methylation regions. Specifically, if for some state, if it has low methylation probability on
cell type 1 and high methylation probability on cell type 2 (or the other way round), then it indicates differential methylation states.

We use a binomial hidden Markov model to model the joint probability of the sequence. In particular, for single-sequence model,
given each state $h$ and methylation
coverage $c$, the methylation count $m$ is drawn from a binomial distribution $\bin(m,p_h)$, where $p_h \in [0,1]$ is the expected methylation
probability over state $h$. The multiple sequence model is a generalization over single-sequence model: specifically, the methylation counts over the cells
$(m^1, \ldots, m^r)$ are conditionally independent given $h$, and $m^i | h \sim \bin(m,p_h)$.

Challenges:
\begin{enumerate}
\item We are given a large dataset, and biologists would like to use the full dataset for accurate statistical estimation. Traditional methods such as EM takes
a long time.
\item Another choice is the spectral algorithm based on tensor decomposition~\cite{AGHKT12}. Unfortunately a naive application
of the algorithm does not apply in this context, since we have a large observation spaces.
Computing the full co-occurence matrix/tensor has large time and space complexities.
\item Recently tensor decomposition methods have been generalized to incorporate kernel methods~\cite{SADX14}. However the time complexity of running such algorithm will be at least
$\Omega(n^2)$, where $n$ is the size of the training data.
\end{enumerate}

In this paper, we propose a feature map-based framework tailored to the problem. Furthermore we develop several feature maps, including binning map and beta-Bernoulli
feature map. After recovering the expected feature map, we use a novel recovery procedure, computing the expected methylation probability.

We observe model mismatch in the dataset, thus making the recovered value of transition matrix and initial probability unstable. We introduce a novel stablization
 procedure to address this issue.

In our experiments, we observe comparable recovery results to EM. However, our algorithm has a large improvement in running time.
